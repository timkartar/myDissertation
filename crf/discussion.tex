\section{Discussion}
In this chapter we applied mean field Bayesian Variational inference to
design a network layer for PNAbind which results in smoother binding site predictions on protein surfaces. We also designed a smoothness metric appropriate for the task of protein surface segmentation.

In the PNAbind framework, the model predicts whether a protein would bind nucleic acids or not and segments the protein surface into binding and non-binding regions. However, it does not try to predict binding specificity (i.e. what nucleic acid squence is preffered) and is only based on protein surface and physicochemical features. I.e. we assume there are some commonalities between the proteins binding to, say, ss-DNA (for PDNA-74) and this model implicitly learns those sets of commonalities. 

The PNAbind package has been published as joint work led by Jared Sagendorf, who mentored me in this project, with contributions from Jiawei Hunag, Prof. Xiaojiang Chen and supervised by Prof. Remo Rohs\citep{Sagendorf2024}. In recent years multiple works have been published which compete with PNAbind \citep{gainza2020deciphering, gligorijevic2021structure, yuan2022alphafold2, xia2021graphbind, tubiana2022scannet, krapp2023pesto, li2023geobind, sverrisson2021fast}. But, no deep learning method to predict binding specificity across protein families has been achieved yet.

As a next step, we work towards predicting binding specificity. One of the key challenges in this problem seting is data sparsity. In the next chapter, we present DeepPBS, a model for protein-DNA  binding specificity prediction.  
