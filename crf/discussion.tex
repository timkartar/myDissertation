\section{Discussion}
In Chapter 1, we used a technique for autoencoding Bayesian Variational inference to model gene
expression time-series data. In this chapter we applied mean field Bayesian Variational inference to
design a network layer for Geobind which results into smoother binding site predictions on protein
surfaces. 

Now, we want to bring light to one key aspect of Geobind. In this framework, the model input has no
explicit information regarding the binding element (e.g. DNA/RNA/drug molecule etc.) and only based on
protein surface and physicochemical features. I.e. we assume there are some commonalities between the proteins binding to
say, ss-DNA (for PDNA-74) and this model implicitly learns those set of commonalities.

However, as a next step we would probably like to bring in the binding elements in the
picture. Normally in binding site prediction tasks this is problematic because including binding
element information makes the
datasets really sparse and unfit for a machine learning task. However, there is a way out of this,
if we move to Generative setting instead of the Discriminative setting of binding site prediction.

In the next chapter, we propose a generalized generative model for  binding element design where we
start from where Geobind's application ends. In this proposal, Given a binding site on a protein we
would like to generate binding elements that could bind to the given site. The binding elements
can be drug molecules or nucleic acid motifs. We shall discuss already performed research works
using generative modeling for novel drug molecule, nucleic acid sequence generation and based upon
insights from them present our proposed model for generalized binding element design.
