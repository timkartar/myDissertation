
%SetFonts


%\date{}							% Activate to display a given date or no date

\section*{Detailed calculation of the CCRF layer update}
Starting from the mean field variational inference result in eq. \ref{mean_field_VI_result} we have, %\red{(Kr\"{a}henb\"{u}hl, 2012)} we have,
\label{crf_detailed}
\begin{equation}
Q_i (H_i) \simeq \frac{1}{Z} \exp \left ( - \alpha || H_i - B_i ||^2  - \beta \sum_{j \in \mathcal{N}(i)} g_{ij} \int_{-\infty}^{\infty} Q(H_j) ||H_i - H_j ||^2 d H_j\right )
\end{equation}

where $H_i, B_i, H_j \in \mathbb{R}^N$. Evaluating this component-wise and using the definition of the $\ell_2$ norm, we can write the energy $E(H_i)$ as

\begin{equation}
E(H_i) = \alpha \sum_n^N (H_{in} - B_{in})^2 + \beta \sum_{j \in \mathcal{N}(i)} g_{ij} \sum_n^N \int_{-\infty}^{\infty} \cdots \int_{-\infty}^{\infty} Q(H_j) (H_{in} - H_{jn})^2 d H_{j1}\cdots dH_{jN}
\end{equation}.

Focusing on the inner most summation of second term, since each component of $H_i$ and $H_j$ is
independent (by eq. 10 of \citet{gao2019conditional}), we can write $Q(H_j) = \Pi_{k=1}^N Q_{jk}(H_{jk})$ and assume that each $Q_{jk}$ is individually normalized. The summation then becomes

\begin{multline}
 \sum_n^N \int_{-\infty}^{\infty}  Q(H_j) (H_{in} - H_{jn})^2 d H_{j1}\cdots dH_{jN} \\
= \sum_n^N \int_{-\infty}^{\infty} (H_{in} - H_{jn})^2  \Pi_{k=1}^N Q_{jk}(H_{jk}) d H_{jk} \\
= \sum_n^N \int_{-\infty}^{\infty} (H_{in} - H_{jn})^2  Q_{jn}(H_{jn}) d H_{jn}\int_{-\infty}^{\infty}  \Pi_{k\neq n}^N Q_{jk}(H_{jk}) d H_{jk} \\
\end{multline} 

where the last step follows because all the $Q_{jk}$'s integrate to 1. Therefore, we have that the energy is given by

\begin{equation}
E(H_i) = \sum_n^N \left [ \alpha (H_{in} - B_{in})^2 + \beta \sum_{j \in \mathcal{N}(i)} g_{ij}  \int_{-\infty}^{\infty} (H_{in} - H_{jn})^2  Q_{jn}(H_{jn}) d H_{jn} \right ]
\end{equation}

Taking the derivative of $E(H_i)$ with respect to the $k$ component of $H_i$ we have

\begin{multline}
\frac{\partial E(H_i)} {\partial H_{ik}} = 2\alpha (H_{ik} - B_{ik}) + \beta \sum_{j \in \mathcal{N}(i)} g_{ij} \int_{-\infty}^{\infty} 2 (H_{ik} - H_{jk})  Q_{jn}(H_{jk}) d H_{jk} \\
= 2\alpha (H_{ik} - B_{ik}) + 2\beta \sum_{j \in \mathcal{N}(i)} g_{ij}  (H_{ik} - \int_{-\infty}^{\infty} H_{jk} Q_{jk}(H_{jk})d H_{jk})
\end{multline}

setting this equal to zero and solving for $H_i$ we have

\begin{equation}
H_{ik}^* = \frac{\alpha B_{ik} + \beta  \sum\limits_{j \in \mathcal{N}(i)} g_{ij} \mathbb{E}_{ H_{jk} \sim Q_{jk} } [H_{jk}]} {\alpha +  \beta  \sum \limits_{j \in \mathcal{N}(i)} g_{ij} }
\end{equation}
%%%%%%% Raktim's section %%%%%%%%%%%%%%%%
\section*{Proposed Algorithm}
Given initial states $B_i$ of the nodes, $H_i^0 = B_i$ maximizes $Q_i^0 = \frac{1}{Z_i^0}exp(-c||H_i^0 - B_i||^2)$. Now, we can use the above update equation and approximately     compute (at $t$ th iteration) $H_i^{t+1}$ (maximising $Q_i^{t+1})$ using $H_i^t$ and so forth until convergence (i.e. upto some iteration K, after which $Q_i^t$ and $Q_i^{t+1}$ are not very different and hence, so are $H_i^{T}$ and $H_i^{T+1}$).
\\
Update for iteration k:
\begin{equation}
 H_{i}^{t+1} = \frac{\alpha B_{i} + \beta  \sum\limits_{j \in \mathcal{N}(i)} (g_{ij} H_j^t)} {\alpha +  \beta  \sum \limits_{j \in \mathcal{N}(i)} g_{ij} }
\end{equation}

We can get the aggregated values $(\sum \limits_{j \in \mathcal{N}(i)} (g_{ij} H_j^t),\sum \limits_{j \in \mathcal{N}(i)} g_{ij})$ for $i$th node in a message passing step using $E, B_i$ and $H_i^t$ $\forall i$.
%\begin{pmialgorithm}[0.9\textwidth]{H}{ Mean Field CCRF Layer}\vskip-2ex
%	\label{algo:tk-means}
%	\begin{algorithmic}[1]
%		\REQUIRE  $B_i$  $\forall i$, $E$ (adjacency information)
%		\STATE Initialize $H_i^0 = B_i$ $\forall i$\COMMENT{$H_i^0$ maximizes $Q_i^0 = \frac{1}{Z_i^0}exp(-c||H_i^0 - B_i||^2)$}
%		\FOR[$T$ signifies convergence]{$t=0,1,2,...,T-1$} 
%        \STATE compute $(\sum \limits_{j \in \mathcal{N}(i)} (g_{ij} H_j^t),\sum \limits_{j \in \mathcal{N}(i)} g_{ij})$ \COMMENT{message passing}
%        \STATE $H_i^{t'} = \alpha B_{i} + \beta  \sum\limits_{j \in \mathcal{N}(i)} (g_{ij} H_j^t)$ 
%        \STATE $H_i^{t+1} = H_i^{t'} / (\alpha +  \beta  \sum \limits_{j \in \mathcal{N}(i)} g_{ij} )$ 
%		\ENDFOR 
%		\STATE $H_i^* = H_i^T$
%		\RETURN $H_i^*$
%	\end{algorithmic}
%\end{pmialgorithm}
%Continued in next page ...
%\newpage
\section*{$Q_i^{(t)}$ and $H_i^{(t)}$ calculation for $t=1,2$}
\begin{align*}
\bE_{H_{jk}^{(0)}}[(H_{ik}^{(1)} - H_{jk}^{(0)})^2] &= \int_{x} exp[-\alpha(x - B_{jk})^2 ](H_{ik}^{(1)}-x)^2dx\\
 &=\int_{t} exp[-\alpha(t - (B_{jk} - H_{ik}^{(1)}))^2)t^2dt\\
 &=Var(T) + \bE[T]^2  \hspace{10pt}\text{($T = X - H_{ik}^{(1)}$)}\\
 &= \bE[T]^2 + const.\\
 &= (H_{ik}^{(1)} - B_{jk})^2\\
 \implies Q_i^{(1)} &= \frac{1}{Z_{i}^{(1)}}exp(-E(H_i^{(1)}))\\
 \implies E(H_i^{(1)}) &= \sum_{k=1}^{N}[\alpha(H_{ik}^{(1)} - B_{ik})^2 + \beta\sum_{j\in\cN(i)}g_{ij}(H_{ik}^{(1)} - B_{jk})^2]\\
 \text{Taking partial derivative and setting to 0:}\\
 H_{ik}^{(1)} &= \frac{\alpha B_{ik} +  \beta\sum_{j\in\cN(i)}g_{ij}B_{jk}}{\alpha + \beta\sum_{j\in\cN(i)}g_{ij}}\\
 ----- & ------\\
 \end{align*}
\begin{align*}
 \bE_{H_{jk}^{(1)}}[(H_{ik}^{(2)} - H_{jk}^{(1)})^2] &= \int_{x} exp[-\alpha(x - B_{jk})^2 - \beta\sum_{p\in\cN(j)}g_{jp}(x - B_{pk})^2](H_{ik}^{(2)}-x)^2dx\\
        &=\int_{t} exp[-\alpha(t - (B_{jk} - H_{ik}^{(2)}))^2 - \beta\sum_{p\in\cN(j)}g_{ip}(t  - (B_{pk} -H_{ik}^{(2)}))^2]t^2dt\\
 &=Var(T) + \bE[T]^2  \hspace{10pt}\text{($T = X - H_{ik}^{(2)}$)}\\
 &= \bE[T]^2 + const.\\
\end{align*}
\newpage
\begin{align*}
\bE[T] &= argmin_{t} (\alpha(t - (B_{jk} - H_{ik}^{(2)}))^2 + \beta\sum_{p\in\cN(j)}g_{jp}(t  - (B_{pk} -H_{ik}^{(2)}))^2)\\
\implies \bE[T] &= \frac{\alpha(B_{jk} - H_{ik}^{(2)}) + \beta\sum_{p\in\cN(j)}g_{jp}(B_{pk} -H_{ik}^{(2)})}{\alpha + \beta\sum_{p\in\cN(j)}g_{jp}} \\
\implies \bE[(H_{ik}^{(2)} - H_{jk}^{(1)})^2] &= \Big{[}\frac{\alpha(B_{jk} - H_{ik}^{(2)}) + \beta\sum_{p\in\cN(j)}g_{jp}(B_{pk} -H_{ik}^{(2)})}{\alpha + \beta\sum_{p\in\cN(j)}g_{jp}}\Big{]}^2\\
&=\Big{[}\frac{\alpha(H_{ik}^{(2)} - B_{jk}) + \beta\sum_{p\in\cN(j)}g_{jp}(H_{ik}^{(2)} - B_{pk})}{\alpha + \beta\sum_{p\in\cN(j)}g_{jp}}\Big{]}^2\\
&= \Big{[}\frac{H_{ik}^{(2)}(\alpha + \beta\sum_{p\in\cN(j)}g_{jp})}{\alpha + \beta\sum_{p\in\cN(j)}g_{jp}} - \frac{\alpha B_{jk} +  \beta\sum_{p\in\cN(j)}g_{jp}B_{pk}}{\alpha + \beta\sum_{p\in\cN(j)}g_{jp}}\Big{]}^2\\
&= \Big{[}H_{ik}^{(2)} - \frac{\alpha B_{jk} +  \beta\sum_{p\in\cN(j)}g_{jp}B_{pk}}{\alpha + \beta\sum_{p   \in\cN(j)}g_{jp}}\Big{]}^2\\
&=\Big{[}H_{ik}^{(2)} - H_{jk}^{(1)}\Big{]}^2\\
 &-----------\\
\end{align*} 
\begin{align*}
 Q_i^{(2)} &= \frac{1}{Z_{i}^{(2)}}exp(-E(H_i^{(2)})) \\
 \implies E(H_i^{(2)}) &= \sum_{k=1}^{N}[\alpha(H_{ik}^{(2)} - B_{ik})^2 + \beta\sum_{j\in\cN(i)}g_{ij}(H_{ik}^{(2)} - H_{jk}^{(1)})^2]\\
 \text{Taking partial derivative and setting to 0:}\\
 H_{ik}^{(2)} &= \frac{\alpha B_{ik} +  \beta\sum_{j\in\cN(i)}g_{ij}H_{jk}^{(1)}}{\alpha + \beta\sum_{j\in\cN(i)}g_{ij}}\\
\end{align*}
Similarly we can show the updates of eq.(7) holds true for t=3,4,...
\begin{center}
 ---------------------------------------
\end{center}

