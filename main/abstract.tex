\begin{abstract}
This dissertation contains an account of my research work during my PhD at University of Southern California. My primary focus has been deciphering protein-DNA interaction with data driven deep learning methods. I present, in chapter 1,my work on the segmentation of protein surfaces into nucleic acid binding and non-binding regions. In Chapter 2, I describe DeepPBS , a geometric deep learning method to predict DNA binding specificity given protein-DNA complexes. DeepPBS acts as a bridge between structure determining and specificity determining experiments. In chapter 3, we design and showcase the RNAscape algorithm and webserver, a geometric mapping method of RNA 3D structures to 2D, which attempts to preserve the three dimensional topology (unlike common secondary structure based visualization methods). Chapter 4 describes an updated DNAproDB database. Through this update we introduce both technical advances and an expansion of features included in the analysis.At the same time we recognized lack of a comprehensive analysis and exploration tool for RNA/protein-RNA structures. Being inspired from RNAscape and DNAproDB, we developed RNAproDB, which is described in Chapter 5. RNAproDB is a modern highly interactive structure exploration tool tailored for the complexity and structural variance of RNA structures. In chapter 6, I present an autoencoder model applicable on gene expression time-series data, to learn a regularized latentspace representation and a generative process. We conclude this thesis by discussing current state of the field of structural biology of protein-nucleic acid complexes and discussing future possibilities.
\end{abstract}