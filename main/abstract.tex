This dissertation proposal contains an account of my research work so far, a proposed next project and discussion about possible future work.
Applying probabilistic machine learning models to solve biological questions has been my primary focus as a researcher. 
In chapter 1, we design and showcase RVAgene: a recurrent variational
autoencoder model applicable on gene expression time-series data, to learn a regularized latent
space representation and a generative process. RVAgene is primarily a visualization tool suitable for biological 
knowledge discovery, while also being suitable for de novo data generation, denoising and a more
efficient alternative to hierarchical Gaussian process based methods to cluster such data. We
analyze one synthetic and two real datasets and demonstrate various properties and aspects of the
model and its potential for unsupervised discovery.  In particular, RVAgene identifies new programs of 
shared gene regulation of \textit{Lox} family genes in response to kidney injury. In
chapter 2, we deal with a different biological problem, using the same probabilistic modeling
technique of Bayesian Variational Inference (VI). Here, we start from a model of nucleic acid binding site
prediction (Geobind) developed by Jared Sagendorf and improve upon it by designing a neural network layer using mean field
VI over a continuous Conditional Random Field model, which makes the predicted binding sites
smoother while improving upon prediction metrics. We also design a smoothness metric
addressing the label imbalance problem associated with the task. In chapter 3, we focus on the fact that the Geobind
model for binding site prediction does not take into account information regarding the binding
elements (drug molecules or nucleic acid sequences). The reason is, for a discriminative model like
Geobind, the datasets become very sparse if we aim to train on datasets specific to nucleic acid
sequences. However, a generative model inspired by and similar to RVAgene can be useful in this
scenario. So, we start where Geobind ends. We propose a generalized binding element design model, where we take binding site
information or some region of interest on a protein surface as input, and generate binding elements which
could bind to the given region of interest on a protein surface. Finally, in chapter 4, we conclude
this dissertation proposal while discussing some long term project ideas for future work.
