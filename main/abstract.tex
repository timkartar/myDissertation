This dissertation contains an account of my research work during my PhD at University of Southern California.
My primary focus has been deciphering protein-DNA interaction with data driven deep learning methods. I present, in chapter 1,
my work on the segmentation of protein surfaces into nucleic acid binding and non-binding regions. This chapter describes a graph neural network layer, which enables smooth prediction of binding site labels on the protein surface. We start from a model of nucleic acid binding site
prediction (PNAbind) developed in the Rohs lab and improve upon it by designing a neural network layer using mean field
VI over a continuous Conditional Random Field model, which makes the predicted binding sites smoother while improving upon prediction metrics. We also design a smoothness metric
addressing the label imbalance problem associated with the task.
In Chapter 2, I describe DeepPBS (Deep predictor of binding specificity), the first of its kind geometric deep learning method developed to predict DNA binding specificity of proteins based on a given protein-DNA complex. DeepPBS acts as a bridge between structure determining (which shows mechanism but not sequence diversity) and specificity determining experiments (which reflects sequences diversity but not mechanism). DeepPBS is applicable on experimentally determined, simulated, predicted or designed complexes, resulting in a broad impact in the domain.
In chapter 3, we design and showcase the RNAscape algorithm and webserver, a geometric mapping method of RNA 3D structures to 2D, which attempts to preserve the three dimensional topology (unlike common secondary structure based visualization methods). RNAscape significantly improves over existing competitors in terms of the mapping quality, visualization and customisability.
Chapter 4 describes an updated DNAproDB database, which was originally implemented by Jared Sagendorf. Through this update we
introduce both technical advances and an expansion of features included in the analysis.
DNAproDB is now automatically updated weekly with newly released structures and thereby will
remain up to date as new DNA–protein structures are solved. We also include much larger
complexes, expand external annotations and upload/download formats, and improve the user
experience through a re-organization of the web interface and more visualization options and
controls. We added the annotation of water-mediated hydrogen bonds as a new feature.
At the same time we recognized lack of a comprehensive analysis and exploration tool for RNA/protein-RNA structures. Being inspired from RNAscape and DNAproDB, we developed RNAproDB, which is described in Chapter 5. RNAproDB is a modern highly interactive structure exploration tool tailored for the complexity and structural variance of RNA structures. This is achieved by intricate interplay of a 3D viewer, interface explorer, sequence viewer, secondary structure selector and tabular data:  making it the most versatile tool for analyzing and exploring protien-NA complexes. With the advent of complex structure prediction methods like AlphaFold3, we expect RNAproDB to serve a crucial role in analyzing predicted structures and advance the understanding of cellular biology.
In chapter 6, I present an autoencoder model applicable on gene expression time-series data, to learn a regularized latent
space representation and a generative process. RVAgene is primarily a visualization tool suitable for biological
knowledge discovery, while also being suitable for de novo data generation, denoising and a more
efficient alternative to hierarchical Gaussian process based methods to cluster such data. We
analyze one synthetic and two real datasets and demonstrate various properties and aspects of the
model and its potential for unsupervised discovery.  In particular, RVAgene identifies new programs of
shared gene regulation of \textit{Lox} family genes in response to kidney injury. We conclude this thesis by discussing current state of the field of structural biology of protein-nucleic acid complexes and discussing future possibilities.
