%% CHANGE NAMES AND WRITING

I offer my utmost thanks to my advisor Prof. Remo Rohs, whose encouragement, patience, intelligent decisions and immense knowledge supported me throughout my Ph.D., made it a wonderful experience, and helped me overcome many difficulties.

I am also extremely thankful to my dissertation committee members Prof. Helen M. Berman, Prof. Fengzhu Sun, Prof. Adam L. MacLean and Prof. Aiichiro Nakano, and qualifying exam committee member Prof. Xiaojiang Chen. Their insightful comments enriched and widened my research from various perspectives. 

I joined the QCB department at USC as a PhD student in August 2019. The rigorous curriculum of the CBB PhD program helped me shape my scientific vision and research skills. Soon, the COVID-19 pandemic changed everything about our life and uncertainty covered the world. Thankfully, due to wise and loving care of the department, my advisor Prof. Remo Rohs and other members of the QCB department, I was able to continue my work through the difficult times. Prof. Adam MacLean worked hard to help me oublish my first first-author paper during this time. The pandemic affected scientific travel opportunities severely, during those initial years of my PhD. However, later on I was fortunate to be able to travel to numerous high quality conferences and present my work there. I am thankful to my advisor Prof. Remo Rohs for providing me with these opportunities. In fact, meeting Nobel laureate Prof. Ada Yonath (for solving the structure of ribosome \citep{schluenzen2000structure, harms2001high}) has been one of the most memorable experiences of my life. Traveling to conferences and presenting my work through posters and oral presentation opened up a lot of career and collaboration opportunities for me. My scientific thinking has been shaped by these experiences, coupled with professional growth. I am grateful to the Rohs Lab members for accompanying and participating alongside me in these events and for always being supportive and encouraging.

I also present my gratitude to my collaborator Dr. Cameron Glasscock from the David Baker lab at University of Washington, and to Prof. Ada Yonath and her lab members, for being supportive to my research. 

I am thankful to the Andrew Viterbi Fellowship in Computational Biology and Bioinformatics for supporting me over three years of my PhD.

My sincere thanks goes to the QCB department for allowing me to participate in various departmental activities, which helped shape me as a person. It has been a pleasure to work with and receive all forms of support from Rokas, Tanya, Katie and Christian.

I thank my fellow labmates Yibei, Tsu-pei, Jinsen, Ari, Jesse, Yingfei, George, previous lab members Brendon and Jared. This dissertation would not have been possible without their stimulating discussions and contributions, both in science and in life. Specially, without Jared's mentorship during the earlier years of my PhD, my dissertation would not be able to achieve its current form. I also thank my graduate mentees Zijin, Wei Yu, Chan, Lexi (and others), and my undergraduate mentees Andrew, Hirad, Avinash and Irika. It has been a pleasure to work with all of you and to see your scientific growth. 
Special thanks go to my friends Bryan, Meilu, Sophia, Vivian, Fred, Eric, Priyanka, Chloe, Nic and Anik, for supporting me along the way. You have made my life in Los Angeles a pristine remembrance. I also offer my gratitude to Prof. Lina Bahn, Christine Lee, Yue Qian, and Haesol Lee, for helping me pursue my interest in violin and music, which helped me conquer many challenging times.

Last but not the least, I am forever indebted to my parents without whose decades long hard work and dedication I would not even be here. On the same thread, I am grateful for the support of my extended family members and childhood friends and teachers, who provided me with unfailing support and continuous encouragement throughout the journey.