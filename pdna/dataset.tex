\section{Dataset}
Lack of exisiting standardized dataset made it nexessary for us to create our own dataset. We relied
on the PDB as our source of structure data and on JASPAR and HOCOMOCO as source
of specificity data. These two databases were chosen based on their presentation, ease of access and
comprehensiveness. JASPAR \citep{fornes2020jaspar} catalogues a comprehensive set of experimental
binding specifity data for proteins from various species obtained through various kinds experiments,
while HOCOMOCO \citep{kulakovskiy2018hocomoco} consists of mainly ChIP-seq experiment data for Human
and Mouse proteins. For a detailed distribution of species and experimental diversity of these two
databases refer to \red{supplementary figure on data distribution}. 
\par
Next we looked for protein-DNA cocrystal data available on the PDB for each PWM available to us
using corresponding Uniprot IDs. For Heterodimer PWMs, only co-crystals which has both heteromers
present were kept. We employ DSSR \citep{lu2015dssr} to check for existence of DNA
double helix and annotate in these structures. For our application we stick to double stranded DNA
only, structures not conforming to this requirement were discarded. Moreover base-pair modifications
were replaced by their parent base identity. This results into 1155 filtered PDB chain IDs.
\par
Next we cluster these PDB chains using CD-HIT \citep{fu2012cd} with a 40\% sequence similarity
threshold for clustering. These results into 189 clusters. This step is necessary to ensure our
training dataset is not overrepresenting any particular kind of protein sequence. Next we sample
upto 5 members form each cluster and this results into 595 PDB chain IDs. We split this set
of structures  into 5 cross validation based on the clusters, i.e. the same cluster can either be in
training or validation set for each fold, but not both. This
ensures the trained model doesn't get opportunities for easy transfer of predictions from training
set to validation set and it has to actually the underlying data distribution to predict correctly
for the validation set.
\par
For each structure entry in the final dataset the corresponding PWM was paired with it, if a PWM
existed in both JASPAR and HOCOMOCO, one was randomly chosen. The two ends of each PWM were trimmed to
remove uninformative regions with a 0.5 Information Content (IC) threshold. For each structure, the
corresponding PWM was aligned to the DNA helix using an ungapped local alignment based on IC
weighted PCC scoring function (\red{refer to supp}). This annotates the region on the DNA helix
where predictions should be made and during training, loss computed on.
