\section{Introduction} 

The scope of this work revolves around building a computational model which can analyze a given
protein-DNA structure and make a prediction about the experimental DNA binding specificity. Ideally we
would like to predict the free energey change of the binding process itself. However to train a
machine learning model to do that task would require a huge amount of protein-DNA
structure data annotated with corresponding energy. Also for the same protein it would be necessary
to have multiple DNA sequences bound to it. However, not only do we not have such energy information
but also we rarely have a protein structure bound to more than a couple different DNA sequences on
the PDB \citep{berman2000protein}. However, we do have a reasonable amount of experimental
specificity data in terms of Position Weight Matrices (PWM) on databases like
JASPAR \citep{fornes2020jaspar}, HOCOMOCO \citep{kulakovskiy2018hocomoco} etc. This makes the problem of predicting PWM from
a given protein-DNA co-crystal from PDB a tractable one, which is the question we attempt to solve
in this work using a combination of Convolutional and Graph neural networks.

However, we must first discuss the breadth of existing work which attempt to predict DNA binding
specificty of proteins in various forms and in varied scopes. Predicting DNA binding specificity
of a given protein is one of the toughest challenges of structural biology research. Existing works 
try to tackle this problem in a family specific manner, generally targeting families with abundant data 
and simple binding mechanism. For example, Zinc Finger proteins have been a popular target for such attempts
\citep{persikov2009predicting,molparia2010zif, persikov2014novo,
meseguer2020prediction,aizenshtein2022deepzf}. \red{expand mechanism and details on individual works}
Homeodomains are another family that has received some attention in this regard
\citep{noyes2008analysis,christensen2012recognition,wetzel2022learning}. \red{expand mechanism and
details on individual work}.
\par
\par
\par
