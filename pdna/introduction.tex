\section{Introduction} 

The scope of this work revolves around building a computational model which can analyze a given
protein-DNA structure and make a prediction about the experimental DNA binding specificity. Ideally we
would like to predict the free energey change of the binding process itself. But to train a
machine learning model to do that task would require a huge amount of protein-DNA
structure data annotated with corresponding energy. Also for the same protein it would be necessary
to have multiple DNA sequences bound to it. Unfortunately, not only do we not have such energy information
but also we rarely have a protein structure bound to more than a couple different DNA sequences on
the PDB \citep{berman2000protein}. However, we do have a reasonable amount of experimental
specificity data in terms of Position Weight Matrices (PWM) on databases like
JASPAR \citep{fornes2020jaspar}, HOCOMOCO \citep{kulakovskiy2018hocomoco} etc. This makes the problem of predicting PWM from
a given protein-DNA co-crystal from PDB a tractable one, which is the question we attempt to solve
in this work using a combination of Convolutional and Graph neural networks.

However, we must first discuss the breadth of existing work which attempt to predict DNA binding
specificity of proteins in various forms and in varied scopes. Predicting DNA binding specificity
of a given protein is one of the toughest challenges of structural biology research. Existing works 
try to tackle this problem in a family specific manner, generally targeting families with abundant data 
and simple binding mechanism. For example, Zinc Finger proteins have been a popular target for such attempts
\citep{persikov2009predicting,molparia2010zif, persikov2014novo,
meseguer2020prediction,aizenshtein2022deepzf}. \red{expand mechanism and details on individual works}
Homeodomains are another family that has received some attention in this regard
\citep{noyes2008analysis, alleyne2009predicting, christensen2012recognition,wetzel2022learning}. \red{expand mechanism and
details on individual work}.
\par
Despite all these attempts, a computational model for predicting PWM for a given protein across
families remains largely unsolved, and looking into the problem a little bit gives some pointers on
why so. Protein-DNA binding is often mediated by conformational changes of the protein structure in presence of suitable
DNA. Binding of TBX5 to DNA is one such example. \citet{stirnimann2010structural} shows in
presence of DNA, the binding domain of TBX5 undergoes a change from $\alpha$-helix to $3_{10}$-helix
which is energetically unfavorable. However this allows two phenylalanine residues (F232, F236) on
the protein to form strong stacking interactions with two pairs of Deoxyribose sugars on the DNA
backbone. Predicting such kind of an event from the protein sequence features only or even the
protein structure can be extremely hard. Moreover, complementarily, the shape of the bound DNA
also influences energetics of protein DNA binding \citep{rohs2009role, dror2014covariation}. For example, the global shape of TATA-box binding
proteins make it necessary for DNA helix axis to bend upto $56^{\circ}$ \citep{kim1993crystal}. In fact most contacts
($~75\%$) on protein surface are DNA backbone contacts in various forms and suitable DNA shape
reflected on protein
surface geomentry plays important role in determining many of them. In theory, this should be
something that a computational model might be able to capture. But characterizing global shape of
proteins to a degree of nuance that allows such prediction is not yet achievable. On the other hand,
it should be noted that characterizing DNA shape is a well established process
(\citet{zhou2013dnashape} (shape from sequence), \citet{blanchet2011curves+,lu20033dna} (shape from
structure) ). These observations
portray a picture that protein-DNA binding is a very intimate process of the two actors involved and
fully characterizing the outcomes with the information of protein only, might even be near
unsolvable with the current state of technology. We should note that, one can perform
of homology based K-nearest neighobour search and predict a PWM based on known 
PWMs of proteins with similar sequences (using naive averaging or some form learning, e.g. \citet{schroder2010predicting}). 
But this process is uninteresting to us because this involves no attempt at biophysical modeling of the system in question and thus has low promise of
generating any meaningful knowledge on top of the usual drawback of search based methods being
generally quite slow.
\par
Although the above discussion portrays quite a bleak picture to any enthusiast who is interested in
the problem, it simultaneously hints towards the fact that protein-DNA structure data can be a more
interesting starting point for predicting a PWM. Although, the structure is missing dynamics but it
has other crucial informations necessary for determining specificity including but not limited to
shape of protein and DNA in the bound context, one bound sequence and corresponding chemical
interactions. In this work we build a deep learning model to predict experimentally determined PWMs
from such data and we hope this work serves as a stepping stone towards solving the full problem of
determining DNA  binding specificity of a given protein as discussed above.
\par
