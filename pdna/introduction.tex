\section{Introduction} 

Transcription factors play critical roles in various regulatory functions that are essential to all aspects of life1. Therefore, understanding the mechanisms by which proteins target specific DNA sequences is crucial2. Extensive research has uncovered myriad binding mechanisms that lead to specific high-affinity binding, including strong electrostatic interaction of arginine residues in the DNA minor groove3, deoxyribose sugar-phenylalanine stacking4, bidentate hydrogen bonds (H-bonds) between guanine (G) and arginine (Arg) in the major groove5, and other interactions6,7,8.
\par
Protein-DNA structures are typically9 obtained through X-ray crystallography, nuclear magnetic resonance spectroscopy or cryo-electron microscopy experiments and stored in the Protein Data Bank (PDB)10. Generally, these structures display one bound DNA sequence and the associated physicochemical interactions6 but do not encompass the full range of potentially bound DNA sequences. Conversely, this information can be experimentally obtained through protein-binding microarray11, systematic evolution of ligands by exponential enrichment combined with high-throughput sequencing (SELEX-seq)12, chromatin immunoprecipitation followed by sequencing13, high-throughput SELEX14 or related high-throughput approaches15. These experiments capture the range of possible bound DNA sequences but do not necessarily provide structural information. In essence, these sets of experiments are complementary, and manual examination is often required to correlate molecular interaction details from structural data with binding specificity data6.
\par
Predicting binding specificity for a given protein sequence, across protein families, remains a challenging and unsolved problem, despite progress for specific protein families16,17,18,19,20,21,22,23. Structural changes in the context of binding, along with large mechanistic diversity, contribute to the difficulty15,24. Protein-DNA structures contain valuable information that artificial intelligence can leverage to achieve generalizability across protein families. In this framework, we introduce Deep Predictor of Binding Specificity (DeepPBS). This deep-learning model is designed to capture the physicochemical and geometric contexts of protein-DNA interactions to predict binding specificity, represented as a position weight matrix (PWM)25 based on a given protein-DNA structure (Fig. 1a). DeepPBS functions across protein families (Fig. 2) and acts as a bridge between structure-determining and binding specificity-determining experiments.
\par
Input of DeepPBS is not limited to experimental structures (Fig. 1a). The rapid advancement of protein structure prediction methods, including AlphaFold26, OpenFold27 and RoseTTAFold28, along with protein-DNA complex modelers, such as RoseTTAFoldNA (RFNA)29, RoseTTAFold All-Atom30, MELD-DNA31 and AlphaFold3 (ref. 32), have led to an exponential increase in the availability of structural data for analysis. This scenario highlights the growing need for a generalized computational model to analyze protein-DNA structures. We demonstrate how DeepPBS can work in conjunction with structure prediction methods for predicting specificity for proteins without available experimental structures (Fig. 3a-d). In addition, the design of a protein-DNA complex can be improved by optimizing bound DNA using DeepPBS feedback (Fig. 3e-g). We show that this pipeline is competitive with the recent family-specific model rCLAMPS17 (Fig. 3h,i) while being more generalizable: specifically, DeepPBS is protein family-agnostic, can handle biological assemblies and can predict DNA flanking preferences.
\par
In terms of interpretability, ‘relative importance’ (RI) scores for different heavy atoms in proteins that are involved in interactions with DNA can be extracted from DeepPBS (Fig. 4). As a case study on an important protein for cancer development, we analyze the p53-DNA interface via these RI scores and relate them with existing literature for validation. Additionally, we show that the DeepPBS scores align well with existing knowledge and can be aggregated to produce reasonable agreement with alanine scanning mutagenesis experiments33 (Fig. 4h).
\par
In additional proof-of-principle studies, we apply DeepPBS to in silico-designed protein-DNA complexes targeting specific DNA sequences (Fig. 5), obtained from a recent study that combines structural design with DNA mutagenesis experiments34. Finally, we show that DeepPBS can also be used to analyze molecular simulation trajectories. We demonstrate an example by applying DeepPBS to a molecular dynamics (MD) simulation of Extradenticle (Exd) and Sex combs reduced (Scr) Hox heterodimer in complex with DNA35 with an AlphaFold-based modeled protein linker (Supplementary Section 10, Supplementary Fig. 6 and Supplementary Video 1). DeepPBS is available as a webserver at https://deeppbs.usc.edu.
\par
