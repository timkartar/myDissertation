\section{Introduction} 

Transcription factors play critical roles in various regulatory functions that are essential to all aspects of life \citep{Spitz2012}. Therefore, understanding the mechanisms by which proteins target specific DNA sequences is crucial \citep{Zhao2009}. Extensive research has uncovered myriad binding mechanisms that lead to specific high-affinity binding, including strong electrostatic interaction of arginine residues in the DNA minor groove \citep{rohs2009role}, deoxyribose sugar-phenylalanine stacking \citep{stirnimann2010structural}, bidentate hydrogen bonds (H-bonds) between guanine (G) and arginine (Arg) in the major groove \citep{Helene1977}, and other interactions \citep{Rohs2010,Schildbach1999,Seeman1976}.
\par
Protein-DNA structures are typically \citep{Garvie2001} obtained through X-ray crystallography, nuclear magnetic resonance spectroscopy or cryo-electron microscopy experiments and stored in the Protein Data Bank (PDB) \citep{berman2000protein}. Generally, these structures display one bound DNA sequence and the associated physicochemical interactions6 but do not encompass the full range of potentially bound DNA sequences. Conversely, this information can be experimentally obtained through protein-binding microarray \citep{Berger2009}, systematic evolution of ligands by exponential enrichment combined with high-throughput sequencing (SELEX-seq) \citep{Slattery2011}, chromatin immunoprecipitation followed by sequencing \citep{Park2009}, high-throughput SELEX \citep{Jolma2013} or related high-throughput approaches \citep{Slattery2014}. These experiments capture the range of possible bound DNA sequences but do not necessarily provide structural information. In essence, these sets of experiments are complementary, and manual examination is often required to correlate molecular interaction details from structural data with binding specificity data \citep{rohs2009role}.
\par
Predicting binding specificity for a given protein sequence, across protein families, remains a challenging and unsolved problem, despite progress for specific protein families \citep{persikov2014novo, Wetzel2022, persikov2009predicting, Sofia2022, Meseguer2020, molparia2010zif, christensen2012recognition, Yanover2011}. Structural changes in the context of binding, along with large mechanistic diversity, contribute to the difficulty \citep{Slattery2014, Chiu2023}. Protein-DNA structures contain valuable information, which has been used to come up with models of specificity tested on small datasets \citep{morozov2005protein}. Artificial intelligence can leverage this information to broadly achieve generalizability across protein families. In this framework, we introduce Deep Predictor of Binding Specificity (DeepPBS). This deep-learning model is designed to capture the physicochemical and geometric contexts of protein-DNA interactions to predict binding specificity, represented as a position weight matrix (PWM) \citep{Stormo2013} based on a given protein-DNA structure (Fig. 1a). DeepPBS functions across protein families (Fig. 2) and acts as a bridge between structure-determining and binding specificity-determining experiments. 
\par
Input of DeepPBS is not limited to experimental structures (Fig. 1a). The rapid advancement of protein structure prediction methods, including AlphaFold \citep{Jumper2021}, OpenFold \citep{Ahdritz2024} and RoseTTAFold \citep{Baek2021}, along with protein-DNA complex modelers, such as RoseTTAFoldNA (RFNA) \citep{baek2024na}, RoseTTAFold All-Atom \citep{Krishna2024}, MELD-DNA \citep{Esmaeeli2023} and AlphaFold3 \citep{Abramson2024}, have led to an exponential increase in the availability of structural data for analysis. This scenario highlights the growing need for a generalized computational model to analyze protein-DNA structures. We demonstrate how DeepPBS can work in conjunction with structure prediction methods for predicting specificity for proteins without available experimental structures (Fig. 3a-d). In addition, the design of a protein-DNA complex can be improved by optimizing bound DNA using DeepPBS feedback (Fig. 3e-g). We show that this pipeline is competitive with the recent family-specific model rCLAMPS \citep{Wetzel2022} (Fig. 3h,i) while being more generalizable: specifically, DeepPBS is protein family-agnostic, can handle biological assemblies and can predict DNA flanking preferences.
\par
In terms of interpretability, ‘relative importance’ (RI) scores for different heavy atoms in proteins that are involved in interactions with DNA can be extracted from DeepPBS (Fig. 4). As a case study on an important protein for cancer development, we analyze the p53-DNA interface via these RI scores and relate them with existing literature for validation. Additionally, we show that the DeepPBS scores align well with existing knowledge and can be aggregated to produce reasonable agreement with alanine scanning mutagenesis experiments \citep{Morrison2001} (Fig. 4h).
\par
In additional proof-of-principle studies, we apply DeepPBS to in silico-designed protein-DNA complexes targeting specific DNA sequences (Fig. 5), obtained from a recent study that combines structural design with DNA mutagenesis experiments \citep{Glasscock2023}. Finally, we show that DeepPBS can also be used to analyze molecular simulation trajectories. We demonstrate an example by applying DeepPBS to a molecular dynamics (MD) simulation of Extradenticle (Exd) and Sex combs reduced (Scr) Hox heterodimer in complex with DNA \citep{Joshi2007} with an AlphaFold-based modeled protein linker (Supplementary Section 10, Supplementary Fig. 6 ). DeepPBS is available as a webserver at \url{https://deeppbs.usc.edu}.
\par
