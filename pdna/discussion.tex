\section{Discussion}

Computationally identifying which DNA sequences, a given protein will bind to remains a challenging question. Although proteins from certain DNA-binding families, such as homeodomain \citep{Christensen2012, Dror2014, noyes2008analysis, Wetzel2022} and C2H2 zinc finger proteins \citep{Wetzel2022, Persikov2014, persikov2009predicting, meseguer2020prediction, Persikov2011, persikov2015systematic} have been studied extensively in this regard, a generalized model of binding specificity remains elusive. This complexity emanates, in part, from the pivotal role that the protein and DNA conformation or shape play in the context of binding specificity. For example, TBX5 undergoes an $\alpha$- to $3_{10}$-helix conformational change when interacting with DNA. Despite the energy penalty, this transformation, in conjunction with an appropriately matching DNA shape, instigates a strong phenylalanine-sugar ring stacking, thereby facilitating binding \citep{Stirnimann2010}. Another example is the Trp repressor protein, which exhibits an almost entirely geometry-driven binding specificity. This protein only forms direct and water-mediated H-bonds with the backbone phosphates \citep{Otwinowski1988}, and the DNA shape required for optimal binding gives rise to sequence specificity. Capturing such interactions and how they lead to binding specificity with protein information alone is complicated and cannot be understood in a sequence space alone \citep{Chiu2023, Zhou2015}. Furthermore, for many protein families, the protein monomer is insufficient \citep{Kitayner2006} for binding; a biological assembly, potentially with other interaction partners \citep{Nair2003}, is often necessary.
\par
DeepPBS achieves generality across protein families with the tradeoff of requiring a docked sym-helix, representing a significant step toward solving the larger unsolved problem. As demonstrated in this work, coupling DeepPBS with attempts to model protein–DNA complexes provides a significant step forward in predicting binding specificity across families, based solely on protein information.
\par
DeepPBS allows exploration of exciting future possibilities, including the creation of DNA-targeted protein designs that could potentially contribute to therapeutic advancements. DeepPBS could serve as a preliminary screening tool for devised candidate complexes, ensuring their specificity to the intended target DNA sequence before any costly experimental validations. Moreover, recent studies have shown that transcription factor–DNA binding can energetically favor mismatched base pairs \citep{afek2020dna}. Given the combinatorial complexity of possible hypotheses, deciding which DNA mismatch experiments to perform to discover more such instances poses a significant challenge. Although there is currently a lack of training data for base-pair mismatches, the DeepPBS architecture, in theory, could facilitate the prediction of mismatched base-pair binding specificity. This approach could assist in deciding which experiments to conduct.
\par
In summary, we have introduced a computational framework that distills the intricate structural nuances of protein$-$DNA binding and bridges this understanding with binding specificity data, effectively connecting structure-determining and specificity-determining experiments. The DeepPBS architecture allows inspection of family-specific ‘groove readout’ and ‘shape readout’ patterns and their effects on binding specificity. Although structure prediction methods like RFNA \citep{baek2024na}, MELD-DNA \citep{Esmaeeli2023} and AlphaFold3 \citep{Abramson2024} can predict a complex from given protein and DNA sequences, they cannot provide insights into binding specificity. The development of these computational methods for structure prediction expands the need of an approach like DeepPBS to derive protein–DNA binding specificity. DeepPBS operates on predicted complexes to yield the binding specificity of the system, thereby guiding the further improvement of modeling techniques for protein–DNA complexes. DeepPBS, despite its generality, exhibits performance comparable to the recently described family-specific method rCLAMPS \citep{Wetzel2022}. In addition to modeled complexes for biologically existing systems, DeepPBS is also applicable to in silico synthetically designed proteins that target specific DNA sequences.
\par
DeepPBS-derived RI scores are biologically relevant. They can be aggregated at a protein residue level, aligning with alanine scanning mutagenesis experimental data. Another advantage of DeepPBS is its speed in predicting binding specificity. Specifically, DeepPBS only requires a single forward call through the model (no required database search or multiple sequence alignment computation), making it suitable for high-throughput applications such as analyzing MD simulation trajectories (Supplementary \hyperref[fig:pdnaS6]{Fig. S6}). In this context, DeepPBS is robust to small dynamical fluctuations and can respond to conformational changes (Supplementary Video 1).
\par
The current version of DeepPBS has inherent limitations. It is tailored for double-stranded DNA and is not yet applicable to single-stranded DNA, RNA or chemically modified bases. However, there is potential for extending the model to accommodate these different scenarios as well as other polymer–polymer interactions and potentially for mechanistic mutations. Further limitations include data limitations, as discussed in Supplementary Section 12. The DeepPBS architecture can be refined and expanded in terms of applications and engineering enhancements. Collectively, these possibilities hint at an exciting future for molecular interaction studies and computationally driven synthetic biology. 

\section{Data availability}
Datasets used for all analysis and associated custom scripts were depos-
ited via figshare at \url{https://doi.org/10.6084/m9.figshare.25678053}. Accession codes for discussed structures from the PDB: 1L3L,
7CLI, 2R5Z, 1CIT, 1F4K, 1GJI, 1TC3, 2BSQ, 2C9L, 5ZGN, 1BBX, 1KLN,
1N5Y, 5YUZ, 1QAI, 1XC8, 6T8H, 4TUI, 1DH3, 7OH9 and 1APL. UniProt
accession codes for protein sequences discussed (folded with RFNA):
Q8IUE0, Q6H878, O43680 and Q4H376. Accession codes for discussed
experimental specificity data from JASPAR2022 and HOCOMOCOv11:
MA1897.1, MA1568.1, MA1031.1, MA1572.1, MA0112.2, MA0112.3,
ESR1\_HUMAN.H11MO.0 and NFKB2\_HUMAN.H11MO.0.B. Mutagenesis experiment data used are available from the SAMPDI website
(\url{http://compbio.clemson.edu/media/download/SAMPDI\_dataset}.
xlsx). MELD-DNA modeled complex data were taken from Zenodo
at \url{https://doi.org/10.5281/zenodo.7501937}. Source data are
provided with this paper.

\section{Code availability}
Installable source code, pretrained models, associated guidelines and
various custom scripts can be found via GitHub at \url{https://github.com/
timkartar/DeepPBS}. The implementation is also available via a Code
Ocean capsule at \url{https://doi.org/10.24433/CO.0545023.v2}. In addition,
DeepPBS is accessible as a webserver through \url{https://deeppbs.usc.edu}.
