\section{Introduction} 

Designing binding elements for target regions in protein residues or complexes is a hard
computational problem  which has huge impact in biotechnology and pharmaceutical ventures.
This constitutes of drug design and nucleic acid sequence/position weight matrix (PWM)
design. The processes generally employed in solving these kind of problems are traditionally virtual and
experimental screening of large amount of candidate binding elements. These processes are extremely
computationally/experimentally intensive and costly. 
\par
Recently, generative machine learning models have
been used to make significant advances in such tasks. The key models generative models that are most
used are mainly of two kind: Variational Autoencoders (VAE) \citep{Kingma2014} and its many
variations \citep{higgins2016beta, sohn2015learning,dilokthanakul2016deep}; Generative Adversarial
Networks \citep{goodfellow2014generative} and variations \citep{wang2018high,zhu2017unpaired}.
\citet{gomez2018automatic} first proposed a variational autoencoder model for drug molecule design.
Similar to RVAgene \red{(cite)} this method encodes training drug molecules represented as SMILES
\citep{weininger1988smiles} string in a regularized latent space which then can be sampled and
decoded to generate new candidate molecules. They also train an additional network to optimize the
generative process based on given chemical properties. This initial model sparked a flurry of folow up 
works mainly addressing various aspects of the problem: e.g. enforcing constraints on the generative process 
such that the generated molecules are chemically valid \citet{kusner2017grammar}, increasing diversity of the generated
molecules, conditional generation etc. 
