\section{Discussion}
In this chapter we described our proposal for designing a generative model for design binding
elements given binding site information on protein surface. The proposed model is a variation over the VAE
framework \citep{Kingma2014} for unsupervised representation learning. However, it should be noted,
that the architecture described here is not final can change after further experimentation.

Next, we need to shade a little bit of light on the datasets that can be used to train such a model.
Primary source for all our data will be the protein data bank \citep{berman2000protein}. It should
be noted that, for nucleic acid PWM generation, although the prediction task is PWM generation, the
training data does not PWM information. Computationally sequences and PWMs are equevalent. the model
sees both sequences and PWMs as a vector of dimension $L \times 4$ ($L$ is length of the
sequence/PWM). Only difference, the vectors are one-hot for sequence and can have fractional values
for PWMs i.e PWMs are soft sequences only. Therefore, it is possible to train the model with
sequence data directly provided the data is rich enough and PWM information is not needed in the
training set. 

Another point that should be discussed is concerning the architecture of Encoder and Generator
networks. The encoder network, although performing the same task for both the drug molecule
generation setting and PWM generation setting, may need to have different architectures. This may be necessary because binding sites for drug molecules are much smaller tha
nucleic acid binding sites (which therefore, can also be discontinuous). This would lead to a point
cloud representation being more favourable compared to mesh representation for PWM generation task.
The generator network will have different architectures based upon the way the binding element
generated is represented as discussed earlier in this chapter.

We hope this project becomes a success and we can present it as a powerful computational tool to the
Computational Biology and Bioinformatics community.
