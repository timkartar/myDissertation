\section{Discussion}
In this chapter we described our proposal for designing a generative model for design binding
elements given binding site information on protein surface. The proposed model is a variation over the VAE
framework \citep{Kingma2014} for unsupervised representation learning. However, it should be noted,
that the architecture described here is not final can change after further experimentation.

Next, we need to shade a little bit of light on the datasets that can be used to train such a model.
Primary source for all our data will be the protein data bank \citep{berman2000protein}. It should
be noted that, for nucleic acid PWM generation, although the prediction task is PWM generation, the
training data does not PWM information. Computationally sequences and PWMs are equevalent. the model
sees both sequences and PWMs as a vector of dimension $L \times 4$ ($L$ is length of the
sequence/PWM). Only difference, the vectors are one-hot for sequence and can have fractional values
for PWMs i.e PWMs are soft sequences only. Therefore, it is possible to train the model with
sequence data directly provided the data is rich enough and PWM information is not needed in the
training set. 

We hope this project becomes a success and we can present it as a powerful computational tool to the
Computational Biology and Bioinformatics community.
