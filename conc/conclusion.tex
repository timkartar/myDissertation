%\Chapter[Conclusion]{Conclusion}

I joined the QCB department at USC as a PhD student in August 2019. The rigorous curriculum of the CBB PhD program helped me shape my scientific vision and research skills. Soon, the COVID-19 pandemic changed everything about our life and uncertainty covered the world. Thankfully, due to wise and loving care of the department, my adviser Prof. Remo Rohs and other members of the QCB department, I was able to continue my work through the difficult times. Prof. Adam MacLean worked hard to help me finish the RVAgene project and get it published \citep{Mitra2021}. The pandemic affected scientific travel oppportunities severely, during those intial years of my PhD. However, later on I was fortunate to be able to travel to numerous high quality conferences and present my work there. I am thankful to my advisor Prof. Remo Rohs for providing me with these opportunities. In fact, meeting Nobel laureate Prof. Ada Yonath (for solving the structure of ribosome \citep{schluenzen2000structure, harms2001high}) has been one of the most memorable experience of my life. I am also thankful to prof. Yonath for supporting us on the RNAscape project. Travelling to conferences and presenting my work their through posters and oral presentation opened up a lot of career and collaboration opportunities for me. My scientific thinking has been shaped by this experiences, coupled with professional growth. I am grateful to the Rohs Lab members for accompanying and participating alongside me in these events and for always being supportive and enouraging.


Overall, it has been a fascinating experience to work in the intersection of artificial intelligence and structural biology during these past years. In 2021, we witnessed a more than half a century old problem, protein folding, being almost solved by AlphaFold2 \citep{Jumper2021}, as I was working on my project of modeling protein-DNA binding specificity based on structures. Efforts were immediately underway, to attempt to predict structures of higher order complexes of biomolecules \citep{evans2021protein,baek2024na}, and recently took a big step forward through AlphaFold3 \citep{Abramson2024}. However, these complex structure prediction methods, thus far, are not yet able to model binding specificity. This puts our model, DeepPBS \citep{Mitra2024}, which can look at a predicted or designed complex, and predict binding specificity in a uniquely synergistic position. In my view, combination of structure prediction and specificity prediction methods is the future of predicting/desgining biologically meaningful complexes. In fact, a fresh direction of thinking about this problem is joint modeling of structure and specificity. However, this is still quite ambitious as data sparsity poses a big challenge, especially for complexes involving nucleic acids. 

Alongside my work on protein-DNA, concurrent events, and my advisor's encouragement inspired me to explore the field of RNA biology, which led to the projects, RNAscape and RNAproDB. As of now, the field of RNA structures is also being shaped by artificial intelligence methods \citep{he2024ribonanza}. However, an AlphaFold level breakthrough is still out of reach \citep{schneider2023will}. Recent works have shown progress in protein structure targeted RNA structure design \citep{nori2024rnaflow} and prediction of protein-RNA binding energy \citep{han2024copra}. Although promising, a lot of it is still quite preliminary and/or lacks biological validation. A DeepPBS like model for RNA binding specifcity prediction is also non-existent (although there has been some progress \citep{Lam2019}). It has also been known for a while that solvent molecules have an effect on protein-nucleic acid recognition \citep{Otwinowski1988}. However, the extent of this phenomenon has not been quantified. Several further directions regarding DeepPBS remains to be explored. Both JASPAR and HOCOMOCO released updated versions recently \citep{Rauluseviciute2024,vorontsov2024hocomoco}. Many aspects of DeepPBS could be rethought in the light of ever expanding repertoire of new deep learning techniques \citep{ho2020denoising, anand2022protein}. Training new DeepPBS models on this data may improve performance. Somewhat related, the input to DeepPBS is a static structure. Augmenting the dataset with more conformers (potentially sampled through MD simulation) could improve performance too. Although there are some existing options \citep{van2006information}, a really good method for docking DNA to a target protein structure is still unavailable. Solving this problem might finally lead to a general model of binding specificity, which can operate based on protein structure alone. However, this might be too complex and a family specific docking algorithm might be the first step towards this direction. All these possibilities make me excited to experience what the future holds in store.

Structure, function and localization of non-coding RNA is also something that remains to be studied and modelled. 70-90\% mammalian genome get transcribed into RNA, but only 1\% of it get translated. Rest are known as ncRNA (non coding RNA). lncRNAs are defined by their size range of 200 bases to 10 kilobases. For a long time lncRNAs were regarded as only transcriptional noise. But, recently, it has been shown that they perform important important regulatory functions. Variations in lncRNA expression has been shown to be significantly correlated with certain disease traits \citep{wapinski2011long} and they are tissue-specific \citep{seifuddin2020lncrnakb}. LncRNA expression actually better explains certain cancer
classification data compared to mRNA expression \citep{al2019long}. Recent work on population-scale tissue transcriptomics \citep{de2021population} discovers high tissue specific regulation of lncRNA. They also identify 800 lncRNA-trait relationships which are not explained by protein coding genes. Hence, there is a growing interest in developing computational methods addressing various kinds of biological problems in lncRNAome. \citep{alam2020deep} discusses Deep learning methods recently being developed for
such tasks. Some of these works include, lncRNA-protein interaction prediction
\citep{pan2016ipminer, zhao2018bipartite, yi2018deep, zhan2019bgfe, peng2019rpiter}, lncRNA  identification \citep{baek2018lncrnanet,yang2018lncadeep, tripathi2016deeplnc},
learning regulatory information \citep{alam2019deepcnpp, alam2019deepel}, predicting subcellular localization of lncRNAs \citep{gudenas2018prediction},    lncRNA-miRNA interaction prediction \citep{huang2019predicting}, lncRNA-disease association prediction \citep{hu2019deep, xuan2019dual, al2019long, xuan2019graph} etc. All these tools might improve in future through incorporation of structural and specificity information, as structure and specifcity prediction methods evolve over time.

With the improvements in structure prediction and design, there is an increasing need of high quality analysis tools for the biologists to be able to study, visualize and explore these structures. In later part of my PhD, we built an updated DNAproDB, RNAscape \citep{Mitra2024rnascape} and RNAproDB to address this need. These tools are designed as a webserver for a user to be able to analkze protein-DNA-RNA structures without requiring a programming background. DNAproDB and RNAproDB are also accompanied with a pre-analyzed collection, which a user can explore immediately. The modern, interactive user interface is a key feature of these tools. I believe these tools will be an essential asset of the field to analyze high throughput predicted/designed complexes, as they continue to evolve, in near future. 

I conclude this dissertation by expressing my deepest gratitude, love and friendship to everyone in my life, who supported me through this journey. I hope my scientific contribution helps towards further progress of our society and to improve the human condition in general.
